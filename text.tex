\documentclass[a4paper, 11.5pt]{article}


%%% HEADER INFORMATION

\usepackage[ngerman]{babel}
\usepackage{hyperref}
\usepackage{xltxtra}
\usepackage{listings}
\usepackage{csquotes}


\renewcommand{\baselinestretch}{1.1} 
% other definitions
\usepackage{fontspec}
\usepackage[margin=2.5cm, tmargin=1.3cm]{geometry}
\usepackage[table]{xcolor}
\usepackage[pages=some, placement=top]{background}
\usepackage{tabularx}
\usepackage{colortbl}
\usepackage{graphicx}
\usepackage{environ}

% standard fff colours
\definecolor{fff}{RGB}{27, 115, 64}
\definecolor{bright}{RGB}{28, 168, 71}

% sets the background (i.e. the green line at the top, as well as the logo)
\backgroundsetup{
scale=1,
color=black,
opacity=1,
angle=0,
contents={\includegraphics[width=\paperwidth]{resources/image.pdf}}
}

% no initial indents in paragraphs
\setlength{\parindent}{0cm}

% width of table lines
\setlength\arrayrulewidth{2pt}



% GRAPHICAL ELEMENTS

% define section headers
\renewcommand{\section}[1]{
    \vspace{0.5\baselineskip}
    {\jost\color{fff}\Large #1}
}


\newcommand{\makeheader}{
% enable background
\BgThispage
% suppress page number
% header and title
{\color{white}
{\headline \maintitle}\\
\vspace{1\baselineskip} \\
{\jost\LARGE \description}
} \\

\vspace {1.5\baselineskip} % width of lines in table
}


\NewEnviron{agtable}
{%
    \vspace{0.5\baselineskip}
    \begin{tabularx}{\textwidth}{X!{\color{white}\vrule width 2pt}X}%
    \BODY
    \end{tabularx}
}

% Line of header cells
\newcommand{\tableheader}[2]{
\arrayrulecolor{white}\hline
\jost\large\color{white}\cellcolor{fff} #1 & 
\jost\large\color{white}\cellcolor{fff} #2 \\
}
% line of content cells
\newcommand{\agupdates}[2]{
\color{black}\cellcolor{bright} #1
& \color{black}\cellcolor{bright} #2 \\
}

\newcommand{\calender}[2]{%
\center\color{white}\cellcolor{fff}\jost\Large #1
&
\color{white}\cellcolor{fff}\jost#2
}






% standard fff fonts
\newfontfamily\jost{Jost.ttf}
\newfontfamily\headline[Scale=4.5]{Jost-400.ttf}
\newfontfamily\jostbold{Jost-400.ttf}
\setmainfont[Scale=1]{merriweather.otf}

\newcommand{\maintitle}{W{\small UNSCHLISTE}}
\renewcommand{\description}{Ein offener Brief}

\setlength{\parskip}{\baselineskip}%

\begin{document}

\makeheader


\vspace{\baselineskip}

\subsection*{Sehr geehrte Mitarbeiter*innen der Münchner Staatskanzlei,}

im Hinblick auf das neue Jahrzehnt, in welchem viele für die Zukunft maßgebende Entscheidungen getroffen werden, adressieren wir Ihnen diesen Brief. Dieser Brief ist eine Gedankensammlung - eine Sammlung vieler Wünsche, die Ihnen als Anreiz und Inspiration dienen soll. Dieser Brief ist ein Wunschzettel vieler jungen Menschen, die das zukünftige Fundament der Gesellschaft bilden werden. Er ist der Ausdruck der jungen Generationen, welche nicht unter den schwerwiegenden Folgen von Entscheidungen, für die sie keine Schuld tragen - denn mitbestimmen können sie ja nicht - leiden wollen.

Im Grunde lassen sich alle Wünsche unter einen Satz zusammenfassen: Folgt der Wissenschaft sowie euren eigenen Versprechen und handelt endlich!


Im weiteren Verlauf des Briefes zeigen wir einige Wünsche von Einzelpersonen auf. Es geht uns hierbei nicht darum, unsere Forderungen erneut zu konkretisieren. Es geht darum, jungen Menschen, deren Stimme im politischen Diskurs zu oft übergangen wird, zu Beginn des neuen Jahrzehnt endlich ein Sprachrohr zu geben.


Ein kurzes Überfliegen dieser Liste macht klar: die junge Generationen ist empört! Sie sieht, wie ihre Zukunft bei einer weiteren weiter-so Klimapolitik aussehen wird. Sie hat Angst, wie die Folgen der Klimakrise ihr Leben beeinflussen und bestimmen wird. Und zeitgleich sieht sie, wie die Regierung die Problematik noch nicht ganz ernst nimmt, wie sie die Augen vor der Dringlichkeit des Handelns und dem Ausmaß der Klimakrise schließt. Und das macht sie sauer. 

In fast allen Wünschen geht es daher darum, dass sie sich mehr Handlung und Maßnahmen wünschen- Taten, Politik, weniger Reden, mehr Handeln - das ist nur ein kleiner Ausschnitt von dem, was sich die junge Generation von Menschen mit Einfluss und von denen in Machtpositionen wünscht. 

Auch thematisch zieht alles in eine Richtung: “Klimagerechtigkeit, Climate Action, Energie- und Verkehrswende, ein starkes Klimapaket [...], mehr klimafreundliche Handlungen [...]” - die Liste ist lang und spricht für sich. Das meiste doppelt sich. Die jungen Stimmen sind schockiert, frustriert - und sie haben Angst. Sie haben Angst vor der Irrationalität, mit der ihre Zukunft verbaut wird, vor ihrem kleinen Handlungsspielraum und ihrer Zukunft selbst.

Zwei Wünsche möchten wir an dieser Stelle herausheben. Sie sprechen etwas an, was in unseren Augen dringend zu betonen ist und beschreiben die Lage, in der wir uns gerade befinden, gut:
\vspace{-0.5\baselineskip}
\begin{itemize}
\item[(1)] “Fehler eingestehen, ans Paris[er Abkommen] halten :)” 
\item[(2)] “Eine Reaktion auf uns” 
\end{itemize}
\vspace{-0.5\baselineskip}
Was daran so bemerkenswert ist? Bemerkenswert ist, dass sie die unkomplizierte, nackte Wahrheit aufzeigen: Dezember 2015 setzten sich in Paris fast alle Nationen an einen Tisch und einigten sich darauf, die globale Erwärmung auf unter 2°C , möglichst 1,5°C, zu beschränken. Doch sobald sie von ihren Stühlen aufstanden, und in ihre jeweiligen Länder und Positionen zurückkehrten, kehrten sie auch dem Pariser Abkommen den Rücken zu - sie hielten zwar weiterhin in Worten an den vereinbarten Zielen fest, Handlungen folgten jedoch nur in Form kläglicher Symbolpolitik. 

Jetzt, im Jahre 2020, fordert die junge Generation, dass sich an das Pariser Abkommen gehalten wird und die somit notwendigen Maßnahmen ergriffen werden. Allein dafür wird sie oft als “radikal” oder “idealisitisch” eingestuft. Dabei fordern die jungen Menschen eigentlich nur, dass die Regierung sich an ihre eigenen Ziele hält. Doch diese leugnet lieber die akute Dringlichkeit des Handelns, sie hinterfragt die Umsetzbarkeit von konsequenter Klimapolitik und betreibt stattdessen eine Politik, die das ursprüngliche Ziel mit Absicht verfehlt. 

Aber es ist noch nicht zu spät - der IPCC belegt dies. Aber - und das ist der Grund warum wir gerade diese beiden Wünsche hervorheben möchten - es braucht eine selbstbewusste Bundesregierung, die zu ihren Fehlern steht und sich in Zukunft, gemeinsam mit Wissenschaft und Bevölkerung, rationale Politik macht. Dann geht es auch wieder aufwärts, dann entsteht wieder Vertrauen, dann wächst die Zufriedenheit der Politik gegenüber. Es sind ja nicht nur die Jungen, die sich eine lebenswerte Zukunft wünschen…

Zum Abschluss möchten wir noch mit unserer Definition von Stärke aufhören. Stärke ist unserer Meinung nach nicht, Fehler rhetorisch zu verstecken oder Fakten zu verdrehen. Sie ist auch nicht, Interviewfragen auszuweichen, genauso wenig wie alle Menschen, die nicht die gleiche Meinung teilen in eine Schublade zu stecken, ohne sich auch nur ein einziges Mal zu fragen, ob denn an dem Kritikpunkt nicht doch was dran wäre. Aber - und das ist an dieser Stelle ganz wichtig - Stärke ist auch nicht, keine Fehler zu machen, denn stark ist, wer zu seinen Fehlern stehen kann und aus ihnen lernt, wer sich aufrichtig entschuldigt und wer dann anfängt, die Sache richtig zu machen! Das ist wahre Stärke!

Wir wissen es durchaus zu schätzen, dass Sie diesen offen Brief lesen, ihn in Ihren Reihen diskutieren, genauso wie wir all die Dinge schätzen, die die Bundesregierung gut getan hat. Wir sind nicht gegen die Regierung. Wir sind einfach nur für eine lebenswerte Zukunft für alle - inklusive lebenswertem Planet.


Gehen Sie raus, fragen sie die Profis und handeln sie nach deren Rat - wir werden Ihnen dankbar sein! 

\vspace{\baselineskip}

Mit herzlichsten Grüßen \\

\vspace{-0.5\baselineskip}
Fridays For Future München





\end{document}
